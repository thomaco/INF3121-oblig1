\documentclass{article}

\usepackage[margin=1in]{geometry} % narrow margins
\usepackage[utf8]{inputenc}
\usepackage[T1]{fontenc}


%\usepackage{hyperref}
%\usepackage{listings}
%\usepackage[cm]{fullpage}



\title{Project 1 \\ Report \\ INF4121/3121} % Title
\date{\today} % Date for the report

\begin{document}

\maketitle % Insert the title, author and date

%Requirement 1
\section{Description, analysis and test cases}

\subsection{Description}
We are going to review the program Hangman, which derives its name from the game, written in the language Java. The program consists of 5 source files, namely;
\begin{itemize}
\item Command.java
\item FileReadWriter.java
\item Game.java
\item HangmanTets.java
\item Players.java
\end{itemize}

\textbf{Command.java} only contains an enumerated type, which contains the possible commands for the program. \textbf{FileReadWriter.java} contains the logic for reading from and writing player objects to file for the scoreboard functionality, as well as a method for sorting and printing the scoreboard. \textbf{Game.java} contains the logic for the game itself, which means handling out, user input, checking input and printing the game dialogue. \textbf{HangmanTets.java} only initializes and starts the Game.java logic. \textbf{Players.java} contains the data structure for a player, as well as methods for fetching name and score.

\subsection{Analysis of the testable parts of the program}
Make an analysis of the testable parts of the program that you are using for the project assignment.

\subsection{Design of manual system tests}
Design the manual tests that you would perform for this program, in order to test its func!onality
(system tests). Specify if you used a par4cular test design technique in designing your tests.
Would it make sense or not to write non-func%onal tests for the chosen source code? Mo%vate your
choice.
\subsection{Test cases}
Write a list with the concrete test cases that you obtained, as a result of the analysis and design
performed above. The list needs to contain: pre- and post- test condi*ons (if the case), the test itself, the
expected result and the actual result. Order the list of tests based on the importance of the tests. Can
you tell something about your how did your previous knowledge / experience helped you in crea9ng this
list with tests?

%Requirment 2
\section{Metrics at project and file level}

\subsection{Metrics at project level}

\subsection{Metrics at file level}

%Requirement 3
\section{Improvements based on metrics}

\subsection{Metrics at project level that needs improvment}

\subsection{List of improved/refactored code}
%Both list of code files, and link to github public repo

\subsection{New metrics at project level vs old}

\subsection{New metrics at file level vs old}

\section{Conclusion}
make a couple of remarks about how easy or not it was for you to maintain the code
(to modify it in order to improve it). Is there anything that you would have done differently on
the ini'al code to make its maintenance easier?

\end{document}