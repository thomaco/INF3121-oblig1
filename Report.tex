\documentclass{article}

\usepackage[margin=1in]{geometry} % narrow margins
\usepackage[utf8]{inputenc}
\usepackage[T1]{fontenc}
\usepackage{hyperref}

%\usepackage{hyperref}
%\usepackage{listings}
%\usepackage[cm]{fullpage}



\title{Project 1 \\ Report \\ INF4121/3121} % Title
\date{\today} % Date for the report

\begin{document}

\maketitle % Insert the title, author and date

%Requirement 1
\section{\textit{Requirement 1} - Description, analysis and test cases}

\subsection{Description}
We are going to review the program Hangman, which derives its name from the game, written in the language Java. The program consists of 5 source files, namely;
\begin{itemize}
\item Command.java
\item FileReadWriter.java
\item Game.java
\item HangmanTets.java
\item Players.java
\end{itemize}

\textbf{Command.java} only contains an enumerated type, which contains the possible commands for the program. \textbf{FileReadWriter.java} contains the logic for reading from and writing player objects to file for the scoreboard functionality, as well as a method for sorting and printing the scoreboard. \textbf{Game.java} contains the logic for the game itself, which means handling out, user input, checking input and printing the game dialogue. \textbf{HangmanTets.java} only initializes and starts the Game.java logic. \textbf{Players.java} contains the data structure for a player, as well as methods for fetching name and score.
%end descrition

\subsection{Analysis of the testable parts of the program\newline}
\textit{I.E - what can be tested?\newline}
manual testing = unit, integration,system tesing (Y)
\href{https://en.wikipedia.org/wiki/Manual_testing}{\textbf{Manual testing link}}


%this can be tested:
\subsubsection{White-box testing}
\paragraph{Unit testing}%check functions for each of the files and what they "promise"

%help: http://www.softwaretestingclass.com/white-box-testing/

\begin{itemize}
\item Players.java
\begin{enumerate}
\item
\textit{Player initialization and declaration variables}
\end{enumerate}


\item Game.java
\item FileReadWriter.java
\item Command.java


\begin{itemize}
\item None.
\end{itemize}

\item HangmanTets.java
\begin{itemize}
\item None
\end{itemize}
\end{itemize}


\textbf{Integration tests}



\subsubsection{Black-box testing}
\textbf{System testing\newline}
%Just check if the whole thing run together
%as specified by hangman- we have NO other requirements


\textbf{Regression testing}%applied after changes to code that worked, and checks if it still performs correctly.
%basically apply the same tests as before AFTER changes in code. 

\subsection{Design of manual system tests}


\subsection{Test cases}


%Requirment 2
\section{Metrics at project and file level}

\subsection{Metrics at project level}

\subsection{Metrics at file level}

%Requirement 3
\section{Improvements based on metrics}

\subsection{Metrics at project level that needs improvment}

\subsection{List of improved/refactored code}
%Both list of code files, and link to github public repo

\subsection{New metrics at project level vs old}

\subsection{New metrics at file level vs old}

\section{Conclusion}
make a couple of remarks about how easy or not it was for you to maintain the code
(to modify it in order to improve it). Is there anything that you would have done differently on
the ini'al code to make its maintenance easier?

\end{document}